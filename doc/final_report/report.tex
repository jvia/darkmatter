\documentclass[11pt, a4paper]{report}

\usepackage{graphicx}
\usepackage{a4}
\usepackage{url}
\usepackage{harvard}
\usepackage{wrapfig}
\usepackage{tikz}
\usepackage{gantt}
\usepackage{algorithmic}
\usepackage{algorithm}
\usepackage{paralist}
\usepackage{hyperref}
\hypersetup{
  colorlinks,%
  citecolor=black,%
  filecolor=black,%
  linkcolor=black,%
  urlcolor=black
}


% Report template authored V. Sorge, adapted S. Vickers.


\title{{\normalsize Software Workshop Team Java (06-08165) 2010/11, Dr E. Thompson }\\[2cm]
  Project Report: DarkMatter\\}
\author{Team A2:  \\
  Jeremiah Via \\
  Yukun Wang \\
  Charles Horrell \\
  Joss Greenaway
}

\begin{document}

\maketitle
\chapter*{Work Breakdown}
\label{work-breakdown}

\thispagestyle{empty}

{\small
  %% This is where we detail how we split up the code
  \noindent\begin{tabular}{|l||l|l|l|l|l|}\hline
    \textbf{Coding} & \textbf{Joss (Team Leader)} & \textbf{Jeremiah} & \textbf{Yukun} & \textbf{Charles} \\ \hline\hline
    Documentation   & Contribution: 5\%           & 85\%              & 5\%            & 5\%              \\ \hline
    Unit Tests      & 10\%                        & 30\%              & 30\%           & 30\%             \\ \hline
    Networking      & 0\%                         & 95\%              & 5\%            & 0\%              \\ \hline
    Menu            & 0\%                         & 70\%              & 30\%           & 0\%              \\ \hline
    Animation       & 0\%                         & 10\%              & 90\%           & 0\%              \\ \hline
    Music           & 90\%                        & 10\%              & 0\%            & 0\%              \\ \hline
    AI Player       & 30\%                        & 0\%               & 30\%           & 40\%             \\ \hline
    Human Player    & 0\%                         & 50\%              & 50\%           & 0\%              \\ \hline
    Level Loader    & 0\%                         & 90\%              & 10\%           & 0\%              \\ \hline
    Level Editor    & 0\%                         & 10\%              & 95\%           & 0\%              \\ \hline
    Level Design    & 0\%                         & 33\%              & 33\%           & 33\%             \\ \hline
    Menu Art        & 50\%                        & 0\%               & 50\%           & 0\%              \\ \hline
    Game Art        & 0\%                         & 100\%             & 0\%            & 0\%              \\ \hline
    Physics         & 25\%                        & 25\%              & 25\%           & 25\%             \\ \hline
  \end{tabular}

  \vspace*{1cm}

  %% This is where we detail how we split up the report
  \noindent\begin{tabular}{|l||l|l|l|l|l|}\hline
    \textbf{Report} & \textbf{Joss (Team Leader)} & \textbf{Jeremiah} & \textbf{Yukun} & \textbf{Charles} \\ \hline\hline
    Introduction    & Contribution: 0\%           & 80\%              & 0\%            & 20\%             \\ \hline
    Abstract        & 0\%                         & 100\%             & 0\%            & 0\%              \\ \hline
    Requirements    & 0\%                         & 50\%              & 0\%            & 50\%             \\ \hline
    Design          & 0\%                         & 100\%             & 0\%            & 0\%              \\ \hline
    Validation      & 0\%                         & 100\%             & 0\%            & 0\%              \\ \hline
    Management      & 0\%                         & 40\%              & 0\%            & 60\%             \\ \hline
    Discussion      & 0\%                         & 33\%              & 33\%           & 33\%             \\ \hline
    Conclusion      & 0\%                         & 100\%             & 0\%            & 0\%              \\ \hline
    User Stories    & 0\%                         & 0\%               & 0\%            & 100\%            \\ \hline
    Minutes         & 0\%                         & 0\%               & 0\%            & 100\%            \\ \hline
    \LaTeX          & 0\%                         & 100\%             & 0\%            & 0\%              \\ \hline
  \end{tabular}
}

\tableofcontents
\thispagestyle{empty}

\begin{abstract}
  Games are an interesting experience for a software developer. They
  require vast amounts of effort to implement even the most trivial
  features. They're also unique because we can actually show our work
  to friends and family and get more than ``So?'' as a response. In
  fact, if done correctly, games can inspire excitement in others
  which makes them rewarding to work on.

  Our aim for the module was to extend the premise of an enjoyable
  physics-based game. The inspiration for our game was Osmos by
  Hemisphere Games \cite{osmos}. Osmos is a game with a simple idea:
  absorb motes smaller than you while avoiding motes large than
  you. This game is a casual game that is more about solving puzzles
  than being the quickest.

  We had the idea to change this casual gameplay into something more
  exciting. By making the game multiplayer, the time to carefully
  consider your options is gone. You must now absorb as much matter as
  you can before your opponent does. This game is also unique among
  many multiplayer games because the loss of one player does not
  entail the victory of the other. It is possible for everyone to
  lose!

  Completing the transformation is in the details. We threw away the
  ability to change the speed of time to focus on the competitive
  nature of the game. The goal of the game is to become huge. We
  wanted to create an experience that pressured players to accomplish
  this goal as fast as they could. As a final detail, we added some
  upbeat music instead of the slow, ethereal music of the original.
\end{abstract}


\chapter{Introduction}
\label{cha:introduction}

% Give a brief overview and guide the reader to the important points
% in the following sections.  Tell your reader what you are going to
% tell them. Overview the project highlighting the issues you
% addressed and hinting at how you solved them. The reader should be
% able to decide from the introduction what parts of the report are of
% interest to them

This report aims to document how our team was able to create a
game. We had roughly 10 weeks to create a game from start to
finish. This was a new experience for all of us and we learned many
lessons from it. We are all better programmers now and we have a
better idea of what it means to create software with a team.

We will begin by describing the final specification of our game. It
changed from our initial specification as we learned more about game
programming. This was a new domain for all of us so our initial ideas
of what it meant to make a game morphed as we gained experience in the
field. We will also explain why we chose the features we did and how
those features positively affect the user experience.

After going over the specification, we will go into the design of our
game. This section will go into the vital details of our game
architecture. You will learn about problems we encountered when
implementing features and how we solved them. We want to not only tell
you what architectural choices we made but why we made them. This will
give you insight into our problem solving abilities as a team.

We will then tell you about how we tested our game and validated its
user experience. This is an important section and we aim to highlight
how extreme programming methodologies allowed us to create good
software. We will show you how unit testing allowed us to assert our
intended uses of the code and ensure future feature enhancements would
be prevented from breaking old, stable code.

With testing and validation out of the way, we will talk about how we
managed our project. Because this was our first experience writing
code as a team, it presented us with new opportunities for
learning. We will discuss how extreme programming helped us maintain
forward momentum on our game and how utilising pair programming
ensured that everyone understood the code to a sufficient level.

We would like to wrap our report up with a general discussion of how
the module was for us. Programming a game from start to finish as a
team was a novel experience. As a result of this we made mistakes. We
will talk about the mistakes we made as a team and individually and
the lessons we learned from them. Because of this module we have an
increased maturity in our programming, meaning that we can recognise
the long-term issues with the choices we make.


\section{Initial Ideas}


At our first meeting we considered various games and discussed many
ideas. We debated the merits of various games and ended up deciding on
criterion for a good game. The game we would make would need to lend
itself to an iterative release schedule. This was important so that we
could add new features each week. We wanted to avoid anything that
would require multiple weeks to implement. We also wanted a game that
would be fun to play. This was a challenge because many entertaining
games have a lot of complex features. These two criterion constrained
our search for a game to make.


Ideas for games that were ultimately rejected included:

\begin{description}
\item [Fruit Ninja] A game which involved the slicing of fruit with a
  blade controlled by the mouse. This game was rejected because we
  could add nothing original to it or come up with any particularly
  fun or innovating multiplayer modes other than a who can post the
  highest score style game.

\item [Tower Defence] A game involving the placing of towers and
  weapons to destroy enemies that work their way through a level. This
  could possibly be made multiplayer in a cooperative mode.

\item [Zombie Tower Defence] As above but involving zombies. We
  rejected both the tower defence games as we felt they had been done
  many times before and lacked originality. There were also free
  versions available of the game online that were very playable.

\item [Space Racing Game] A racing game which takes place in space
  where the players must race while avoiding planets and
  asteroids. This was rejected again for lack of originality.
\end{description}


The game we finally settled on was a multiplayer version of Osmos
developed by Hemisphere games \cite{osmos}. We called our game
Darkmatter as we would be setting it in space with the concept of
stars rather than on the cellular level like Osmos.

All of us were happy with this selection as it included features that
played to the strengths of all members of the team. Joss with the
potential to produce exciting graphics and audio effects, Charlie with
the chance to use his physics knowledge and Jeremiah and Yukun to
utilise their strong programming skills.

%%% Local Variables: 
%%% mode: latex
%%% TeX-master: "../report"
%%% End: 

%  LocalWords:  Osmos Darkmatter Joss morphed multiplayer Yukun
  %% DONE(1)

\chapter{Requirements}
\label{cha:requirements}

You have been given only a very general set of requirements; hence
you have to formulate more specific and detailed user requirements
for the applet you chose to do and explain why you chose these
features. Explain your choices carefully (not just lists of bullet
points). Note that not everything about requirements in general is
relevant for this project (for instance, you need not write much
about the hardware requirements, as they are trivial for this
project).
  %% DONE(1)
\include{tex/design}        %% DONE(1)
\chapter{Validation and Testing}
\label{cha:validation}

% Establish what your project can handle successfully, and what its
% limitations are. Use meaningful examples, not lists of trivial
% cases.
%
% Your applet is expected to be in good working order and do something
% useful. This chapter is important, because it describes how you
% assure yourself that that is the case.
% 
% Establish what your project can handle successfully, and what its
% limitations are. Use meaningful examples, not lists of trivial
% cases.

Testing and validation are important aspects of extreme programming
and we did our best to remain faithful to the traditional
approach. Our testing strategy never became very complicated; there
was not enough time to develop a more comprehensive approach. Had
there been time, we would have liked to add integration tests to our
project.

\section{Testing}

Because our team adopted the extreme programming methodoloy, testing
was an important part of our development process. Our pair programming
sessions usually consisted of three stages, the second one being the
creation of unit tests for the code we just wrote. We took advantage
of JUnit in order to create a series of consitent tests. Our aim was
to have as much test coverage as possible. It was not about simply
creating a test for each method but instead ceating a test for each
behavior we intended the method to have. It is for this reason that we
have multiple tests for some methods.

Our testing strategy was quite simple. We created test suites at the
package level and tested all the behaviour we wanted our code to be
able to deal with. This meant identifying potential weak spots in the
code and testing for it. Our process fell short of the test-driven
approach of writing tests first but we still did a good job of
creating tests to make our code fail. We envisioned every way in which
our code could potentially fail and wrote a test for it. This meant
having a lot of failling code at first but as we refined our methods
and assumptions we gradually passed more tests until we passed them
all.

Testing the whole of our project was done in a visual and interactive
way. Part of this was because we did not create integration tests to
test the behaviour of multiple subsystems working
together. Integration tests were outside the scope of our assignment
but we still regret not exploring and implementing them. The other
reason we had to test our game interactively was simply due to the
domain in which we were programming. Games present a challenge for
automated testing. While it would be possible to test some behaviour
with Java's GUI robots, it is a lot harder to put a test into numbers
when we are aiming to test the \emph{feeling} of the game. For these
kinds of tests we simply had to play the game a lot. This was not a
problem, though, as we all enjoyed playing our game. The added benefit
of play-testing the game was that it would excite us to add or enhance
game features.

% Testing is so you can be confident that the software is robust and
% bug-free. What was your strategy for that? How did you plan unit
% testing (for components) and integration testing (for the whole
% applet)? How did the prototype fit in?

% Ideas: 
%
% - We never tested what happens if more players connect to the server
% than there are matter objects on the map.
%

\section{Validation}

Validation was a very informal process for our team. Since we were the
kinds of users our game was intended for, we were able to determine if
the software was headed in the right direction. Because of our weekly
release process, we could review the current state of the game at each
meeting. By each meeting we had coded our respective parts and were
now all together, we could voice opinions on the way the software
worked. This was done multiple times throughout the development of our
game.

Weekly validation was critical to make the zooming and scrooling
features of the game work well. When discussing the feature and how it
worked, we undoubtedly envisioned subtlely different variations as to
how it would look. After a prototype was created we all talked about
it. This input allowed a refinement of the behaviour that we all felt
worked well.

A similar situation happened with the absorption of matter objects. It
took many iterations to get the absroption behaviour exactly
correct. There were many details to consider, such as absorption rate
and change in momentum, whch made working out the exact behaviour we
wanted take a long time. Each week we were able to see a slightly
modified absorption behaviour and by voicing our opinions it reached
its final, excellent state.

% Validation is to check that in the end the applet is useful
% and pleasant to use. What was your strategy for that? Have you tried
% it out with teachers or students? What kind of rolling validation
% did you use for the evolutionary part (developing the GUI)?



%%% Local Variables: 
%%% mode: latex
%%% TeX-master: "../report"
%%% End: 
    %% DONE(1)
\include{tex/management}    %% DONE(1)
\chapter{Discussion}
\label{chap:discussion}

% How did the team operate? What did you do to manage the team?  How
% did you resolve issues in the team?  Did you seek assistance of
% others?  Would you do anything differently in the future?  What is
% your evidence of good or bad team work
%
%
% The documentation of the team management in the report
% shows that the team endeavoured to cooperate and/or when
% issues within the team arose, they were resolved to enable to
% team to function as a coherent unit.
% - Discussion shows that the issues were addressed and provides
% evidence of what might be done differently on future projects.
% - The report acknowledges the assistance of others to resolve
% issues if necessary.
% - A work breakdown is included in the report and is supported by
% other documentation.
%

This was the first major programming project any of us had ever
undertaken in a team. As with all new experiences, we made a number of
mistakes but learnt a great deal from them. This chapter will detail
some of our shortcomings, both personally and as a team, and the
lessons we learned from this whole experience. We are all better
programmers and teammates because of this module and we feel grateful
for this opportunity. We did not follow all the best practises for
working in a team but did an admirable job nonetheless. Each of us
failed in some capacity but we all learnt from it. What follows is the
result of a lot of self-critical reflection. This section is not meant
to give the impression that we were a terrible team, in fact we all
still like each other, just that we made a number of rookie mistakes
which did have \emph{some} impact. Overall we worked well as a team
and we are all proud of the work we created together.

\section{Team Mistakes}

During the development of our game we made a series of missteps which
hindered our development process. None of the mistake were egregious
or even overt at the time but after spending some time to reflect on
the module, we have been able to recognise some ways in which we could
have done better. Among our mistakes were a lack of communication,
ambiguity in responsibility for a particular feature and the failure
of completing a task on time.

We were plagued by communication issues throughout the project. Our
team used Facebook as a means of communication and in retrospect it
was a mistake. Our group Facebook wall is almost entirely populated by
the communication of one person. Facebook is a passive stream of
consumption, i.e., users do not respond to most input they receive on
it. This pattern of use does not lend itself well to a team project
where input from all members is important. As a result of using
Facebook, one person would post questions but never receive
responses. This got in the way of good communication and hurt team
morale. On the next project, alternative means of communication will
be used.

Another mistake was the lack of a rigid break up of work requirements.
This was in part to our agile philosophy of team code ownership and
the responsibility of everyone to work on every class. It, however,
resulted in a lack of understanding about who had to produce an aspect
of the software. It allowed us to be apathetic about the project as a
whole at times, such as week six when the majority of the team
prioritised the Software System Components assignment. In the future,
stricter assignment of work would be beneficial.

It is worth noting that our team leader Joss disappeared midway
through term. Obviously, losing a team member has a negative impact on
a team's ability to complete work. It was also a confusion situation
because we never heard anything from him on the matter. He just
disappeared and we had to find out through his friends that he had
become ill. The main problem from this situation was that our team was
left without a team leader. No one ever took the place of team leader
and so our development happened in a more ad hoc manner. This is
something we should have immediately rectified by chasing him down on
his university email and reassigning the position when it became clear
he would not have been returning.

Our team was also negatively impacted by the failure of some members
to complete their assigned tasks on time. While this was rare, it did
mean that others had to pick up the slack and start working on parts
of the code that they were not meant to be working on. Having to do
another's work prevented some features from making their way into the
game. This is an issue which we should have addressed at the
beginning. We had trouble with this because it is hard to admonish
people we like. We have all learnt the importance of ensuring that
team members complete their task. In future projects, we will all be
more forward about the failures of our teammates to complete their
work.

\section{Individual Mistakes}

\subsection{Charles}

This was my first time working in a team in a software development
environment. While familiar with the processes involved in large team
projects academically, I was less prepared for how this would relate
to a software project. I made quite a few mistakes over the period
which I have managed to learn from.

My main mistake was not maintaining a constant work rate throughout
the term. I prioritised some other modules above this module and
particularly around the time I was doing my communication skills
presentations and essays I was not able to offer the input to the team
java module that I should have. This is a valuable lesson for time
management and work planning. In the future I will plan for these
thing and anticipate it and make it up to my team by doing more work
in advance of the time I am going to be working on other projects.

Another mistake I made was not getting as involved in the production
of software as either Yukun or Jeremiah. While I did write code for
the \texttt{VelocityVector} class, the AI player, unit tests and
produce some levels I did not really write enough code overall. This
was because I took my role as team secretary too literally and perhaps
put too much emphasis of my time on producing team meeting minutes,
weekly reports, user stories and working on the final report than
writing code. In future projects I will work on writing more code.

Another mistake I felt I made was not asking Jeremiah, the most
experienced coder on our team, for help with writing more code. I was
afraid that my inadequacies as a coder would show through and I was
embarrassed by this. I should have utilised the pair programming
sessions more fully in the later weeks of the project as in the first
six weeks to help me code and learn from these two excellent coders. I
learnt from this to always make use of such excellent resources in
future projects.

\subsection{Jeremiah}

While I feel I worked well on a team, I did make a number of mistakes
over the course of our project. These were lack of leadership, the
inability to correctly deal with a teammate's missed deadline and
difficult in sharing code.

In retrospect I should have taken the leadership position of our
team. I had spent the previous term reading books on teamwork and
wanted to apply some of what I had learnt. I had heard of Joss'
reputation for partying and disregard for work and I had hoped to stem
this behaviour. So I applied a technique I had read in a book by
\cite{carnegie1936}: ``Give the other person a fine reputation to live
up to.'' So I figured by making Joss team leader he would live up to
the leadership reputation. It did in fact work for a good portion of
the term. But, unfortunately he disappeared and never gave us any
indication of what had happened to him. I should have taken his role
as soon as a week had passed. This would have prevented our team from
remaining in decision paralysis for son long.

Another mistake I made was not properly dealing with other team
member's uncompleted work correctly. I did not do a good enough job of
chasing them down and making sure they completed their task; getting
the demonstrator involved if nothing had changed. Instead, I would
take it upon myself to finish their task for them. This helps no one
on the team. It delays progress because my time is spent doing other
people's work and it give the other person a pass when they need to
understand that their contribution is important. In the future, I will
make an effort to make sure all my team members contribute evenly.

I also had a hard time sharing code. When programming, I tended to be
pretty controlling about the way the code would function. This made
pair programming sessions frustrating because I wanted to take control
and immediately begin typing what I felt was the correct way to
proceed. Overall, the pair programming sessions did help me with this
feeling. I am now a lot more open minded and patient when it comes to
coding with others.

\subsection{Yukun}

This is my first time to do a team software project. Although the
final result looks pretty good, I still made some mistake in the
process.

The main thing is team work. To be honest, in this project we worked
in a team, but not as a team. The team members worked separately
although at beginning we planned pair programming. And when we lost
the team leader, we did not point out a temporary leader which is one
of the reasons why our project was so slow.

Secondly, I did not have a good communication with the other team
members. My poor English made me ashamed to talk and explain my
ideas. And then since we had the whole term time to finish this
project, I let it fall behind. Especially in the middle term, my
schedule got bogged down which made me less enthusiastic about
coding. Another mistake I made in the project was not fully utilizing
the Junit test. Due to a lack of experience, I always wrote JUnit
tests after programming instead of before. So when I was coding, I was
confused about the target of each method.

In this project, I learned a lot things I never learned before. I
learned how to user version control software by using subversion. And
thanks to Jeremiah, we had Maven which made our work process
convenient. Meanwhile, I learned how to use JUnit test to set goals
for programs. On the other hand, team work is not as simple as I
thought. I should communicate better with team members and when a
problem occurs the team needs to make a decision in time.

%%% Local Variables:
%%% mode: latex
%%% TeX-master: "../report"
%%% End:

% LocalWords:  Facebook Joss hoc another's JUnit

    %% DONE(1)
\include{tex/conclusions}   %% TODO(1)
\appendix
\include{tex/user_stories}  %% DONE(1)
\chapter{Meeting Minutes}
\label{cha:minutes}

\section{Group Meetings}

\subsection*{Meeting 1}
\date{20 January 2011}

\subsubsection{Present}
\begin{enumerate}
\item Joss Greenaway (Team Leader)
\item Charlie Horrell (Secretary)
\item Jeremiah Via
\item Yukun Wang
\end{enumerate}

\subsubsection{Summary}

During this meeting we spent some time getting to know each other and
confirmed the appointment of Joss and myself as officers. The main
content of the meeting was talking about possible game ideas. Ideas
proposed included:
\begin{itemize}
\item An physics game styled on Osmos with added features and
  originality. We decided this was our primary idea and Joss came up
  with the project name of ``Darkmatter'' for the game.
\item An Osmos style racing game using the Osmos propulsion mechanics
  to race blobs around an obstacle course.
\item A tower defense style game possibly involving zombies and a
  cooperative mode.
\item Fruit ninja.
\item Fall down: a two player game involved in a race to see which
  player can fall down quickest.
\end{itemize}


Once we had decided on Osmos as our primary idea and zombie tower
defense as our fall back, we considered ways to put our own spin on
Osmos. These included the addition of weapons to the right click, such
as an antimatter weapon that would split any other dark matter stars
in half or reduce size. And also a weapon to stop the balls in full
flow. The main aspect of originality that we add as well as setting it
in space is the addition of cooperative and competitive game play. We
talked about various modes such as straight up death match style,
i.e., just become the biggest player, to other game modes that would
require players to work together to take down a larger enemy or maybe
a mode where the players are segregated and need to complete tasks
with each other to complete the level.

We also touched on the software engineering principles we would employ
and this included testing methods such as JUnit and the possible use
of pair programming was mentioned as a method as well as other methods
such as ``Scrum''. Jeremiah suggested the use of Maven for our project.

\subsubsection{Action}

\begin{description}
\item[All] Research Osmos and Subversion
\end{description}

%%%%%%%%%%%%%%%%%%%%%%%%%%%%%%%%%%%%%%%%%%%%%%%%%%%%%%%%%%%%%%%%%%%%%

\subsection*{Meeting 2}
\date{25 January 2011}

\subsubsection{Present}
\begin{enumerate}
\item Joss Greenaway (Team Leader)
\item Charlie Horrell (Secretary)
\item Jeremiah Via
\item Yukun Wang
\end{enumerate}

\subsubsection{Summary}

We met for the second time to discuss coding practices and the
structure of our game. We decided that we would leave the start of
coding till week 3 when we knew how to work sockets and networking
into the code as we felt that this would impact our code at every
level. We decided on a pair programming schedule where we would each
code with each other member of the team for an hour a week. We also
decided that we would have this meeting every week. After these
discussions and some discussion of the physics engine we decided to
head to the labs to get Subversion working and test it out. Jeremiah
helped the team with this as he has prior experience of using
subversion.

\subsubsection{Action}
\begin{description}
\item[All] Research coding structure and Osmos. Submit progress logs
  to Subversion.
\item[Charlie] Produce timetable of pair programming and other Team
  Java commitments and upload to Subversion.
\end{description}

%%%%%%%%%%%%%%%%%%%%%%%%%%%%%%%%%%%%%%%%%%%%%%%%%%%%%%%%%%%%%%%%%%%%%

\subsection*{Meeting 3}
\date{1 February 2011}

\subsubsection{Present}
\begin{enumerate}
\item Joss Greenaway (Team Leader)
\item Charlie Horrell (Secretary)
\item Jeremiah Via
\item Yukun Wang
\end{enumerate}


\subsubsection{Summary}

We decided to meet to discuss the impending deadline for the
specification and to make sure we were broadly on the same page
regarding what to put in it. We also talked over software engineering
principles that we were going to use including maven and testing. We
also discussed the GUI and the possible methods of animating the dark
matter. Yukun also showed the team a small model he had made with an
object bouncing around a screen and interacting with the edges. We
then created a sandbox in the subversion repository in which we could
put experimental code to play around with. We decided on physics that
would go into the game that included making the game universe
frictionless to make the modeling easier and to replicate space.

\subsubsection{Action}
\begin{description}
\item[All] Produce specification and submit to Subversion.
\end{description}

%%%%%%%%%%%%%%%%%%%%%%%%%%%%%%%%%%%%%%%%%%%%%%%%%%%%%%%%%%%%%%%%%%%%%

\subsection*{Meeting 4}
\date{1 February 2011}

\subsubsection{Present}
\begin{enumerate}
\item Joss Greenaway (Team Leader)
\item Charlie Horrell (Secretary)
\item Jeremiah Via
\item Yukun Wang
\end{enumerate}


\subsubsection{Summary}

We discussed how we were going to approach the pair programming
sessions over the coming week. It was decided upon that for the
sessions scheduled straight after the meeting that Jeremiah and Joss
would work on the matter class and charlie and Yukun would work on the
physics class. This work was then undertaken and submitted to
subversion. It was raised that charlie and Yukun were not uploading
comments when they submitted work to subversion and this should be
remedied. Jeremiah and Yukun also showed of various prototypes that
they had uploaded to subversion to test out aspects of the game.

\subsubsection{Action}

\begin{description}
\item[Joss \& Jeremiah] Produce Matter Class.
\item[Charlie \& Yukun] Produce Physics Class.
\end{description}

%%%%%%%%%%%%%%%%%%%%%%%%%%%%%%%%%%%%%%%%%%%%%%%%%%%%%%%%%%%%%%%%%%%%%

\subsection*{Meeting 5}
\date{8 February 2011}

\subsubsection{Present}
\begin{enumerate}
\item Joss Greenaway (Team Leader)
\item Charlie Horrell (Secretary)
\item Jeremiah Via
\item Yukun Wang
\end{enumerate}


\subsubsection{Summary}

Firstly we discussed the progress of the previous and team members
were bought up to date on the demonstrator meeting from the previous
week. We discussed the points raised in that meeting and endeavored to
address the issues that came up such as using more scientific English
and producing a detailed weekly report. It was decided that charlie
would produce this in time for the demonstrator meeting for the
following weeks. Jeremiah demonstrated to us the Darkmatter class he
had written and Joss showed us the draft of an opening video for the
game. We then decided on tasks for the week and ranked tasks that
needed to be done in order of priority. The main issues to be
undertaken were programming an AI for the other player in the single
player mode, start programming the networking, fixing the ejection
angles of the balls of matter and introduce an egocentric scrolling
view.

\subsubsection{Action}

\begin{description}
\item[Joss] Fix ejection angles of matter.
\item[Charlie] Produce AI for other player.
\item[Jeremiah] Start the client-server networking.
\item[Yukun] Develop a scrolling system with the player always at the center.
\end{description}

%%%%%%%%%%%%%%%%%%%%%%%%%%%%%%%%%%%%%%%%%%%%%%%%%%%%%%%%%%%%%%%%%%%%%

\subsection*{Meeting 6}
\date{15 February 2011}

\subsubsection{Present}
\begin{enumerate}
\item Joss Greenaway (Team Leader)
\item Charlie Horrell (Secretary)
\item Jeremiah Via
\item Yukun Wang
\end{enumerate}


\subsubsection{Summary}

This was an extremely brief meeting where we just decided that we
would carry on with the previous weeks goals and that Joss would work
on producing the media player. Jeremiah also filled us in on how the
changes to the client-server would effect the rest of the game code
and helped us install the latest versions of maven on our machines.

\subsubsection{Action}
\begin{description}
\item[Joss] Produce media player for the game.
\item[Charlie] Produce AI for other player.
\item[Jeremiah] Start the client-server networking.
\item[Yukun] Develop a scrolling system with the player always at the
  center.
\end{description}

%%%%%%%%%%%%%%%%%%%%%%%%%%%%%%%%%%%%%%%%%%%%%%%%%%%%%%%%%%%%%%%%%%%%%

\subsection*{Meeting 7}
\date{22 February 2011}

\subsubsection{Present}
\begin{enumerate}
\item Charlie Horrell (Secretary)
\item Jeremiah Via
\item Yukun Wang
\end{enumerate}


\subsubsection{Summary}

This was another brief meeting where we again decided to carry on with
the previous weeks work as it was still in progress. Also tidy up and
review JUnit tests for the testing vivas later on in the week.

\subsubsection{Action}
\begin{description}

\item[Joss] Produce media player for game.
\item[Charlie] Produce AI for other player.
\item[Yukun] Develop an scrolling system with the player always at the center.
\item[Jeremiah] Start the client server networking.
\item[All] Review tests.
\end{description}


%%%%%%%%%%%%%%%%%%%%%%%%%%%%%%%%%%%%%%%%%%%%%%%%%%%%%%%%%%%%%%%%%%%%%

\subsection*{Meeting 8}
\date{1 March 2011}

\subsubsection{Present}
\begin{enumerate}
\item Joss Greenaway (Team Leader)
\item Charlie Horrell (Secretary)
\item Jeremiah Via
\item Yukun Wang
\end{enumerate}


\subsubsection{Summary}

This meeting we talked frankly about our disappointing progress of the
previous few weeks and decided on a course of action that would propel
us forward. This involved finishing off previous tasks including
working on finishing networking and making AI smarter and scrolling
smoother. We also decided that a level builder would be a good feature
to improve the games single player mode.

\subsubsection{Action}
\begin{description}

\item[Charlie] Keep working on AI; begin working on final report.
\item[Joss] Continue working on media player.
\item[Jeremiah] Work on networking.
\item[Yukun] Create a level builder.
\end{description}


%%%%%%%%%%%%%%%%%%%%%%%%%%%%%%%%%%%%%%%%%%%%%%%%%%%%%%%%%%%%%%%%%%%%%

\subsection*{Meeting 9}
\date{8 March 2011}

\subsubsection{Present}
\begin{enumerate}
\item Joss Greenaway (Team Leader)
\item Charlie Horrell (Secretary)
\item Jeremiah Via
\item Yukun Wang
\end{enumerate}


\subsubsection{Summary}

This meeting we talked about the things we needed to do to polish off
our project. This included working more on the final report,
integrating the networking and producing a menu to polish up the game.

\subsubsection{Action}

\begin{description}
\item[Jeremiah] Continue working on networking.
\item[Charlie] Work on final report and submit an iteration to Subversion.
\item[Yukun] Work on menu.
\item[Joss] Work on media player.
\end{description}

%%%%%%%%%%%%%%%%%%%%%%%%%%%%%%%%%%%%%%%%%%%%%%%%%%%%%%%%%%%%%%%%%%%%%

% \subsection*{Meeting 10}
% \date{25 January 2011}
%
% \subsubsection{Present}
% \begin{enumerate}
% \item Joss Greenaway (Team Leader)
% \item Charlie Horrell (Secretary)
% \item Jeremiah Via
% \item Yukun Wang
% \end{enumerate}
%
%
% \subsubsection{Summary}
% \subsubsection{Action}


\section{Demonstrator Meetings}
\subsection*{Meeting 1}
\date{20 January 2011}

\subsubsection{Present}
\begin{enumerate}
\item Katrina Samperi (Demonstrator)
\item Joss Greenaway (Team Leader)
\item Charlie Horrell (Secretary)
\item Jeremiah Via
\item Yukun Wang
\end{enumerate}

\subsubsection{Summary}
During this meeting we introduced ourselves to Katrina and pitched our
idea of Darkmatter to her. Katrina approved our idea and informed us of
the basic structure of the module and the importance of
logs,subversion and team work. She also prompted us into choosing a
team name! After a brief discussion about Pokemon we settled on team
``Giant Cow'' because to quote Joss ``one of the new Pokemon looks just
like a giant cow.'' The name was catchy yet suitably absurd and
humorous so ``Giant Cow Games'' was born! We also asked Katrina about the
specification that is due in by the end of week two and it was
suggested to us that we should not go into technical detail but give a
more general idea of what our game will be about and the principles
used to code it. We were also warned about pitfalls previous teams had
had in splitting up components of the project arbitrarily and then
meeting at a later date to merge sections, as this had no fail safe
and often required a lot of extra work to integrate components. We
also raised the issue of possibly using games libraries or other
source code in the programming of the game. Katrina said she would look
into this for us and let us know. Katrina also indicated to us that pair
programming was an excellent idea.

\subsubsection{Action}
\begin{description}
\item[All] Research Osmos and Subversion. Think about content for
  specification.
\item[Charlie] Update Facebook group, minutes and arrange further
  meetings.
\end{description}


%%%%%%%%%%%%%%%%%%%%%%%%%%%%%%%%%%%%%%%%%%%%%%%%%%%%%%%%%%%%%%%%%%%%%

\subsection*{Meeting 2}
\date{7 January 2011}

\subsubsection{Present}
\begin{enumerate}
\item Katrina Samperi (Demonstrator)
\item Joss Greenaway (Team Leader)
\item Charlie Horrell (Secretary)
\item Jeremiah Via
\item Yukun Wang
\end{enumerate}

\subsubsection{Summary}
During this meeting we were given more precise instructions on when to
submit the specification. We were told it had to be in for 12pm on
subversion in PDF format. We also talked about our progress for the
week, including pair programming and testing with maven. Katrina approved
of these measures. We also decided to drop scrum programming partly
because of the roles that different people would have to take on made
the process seem convoluted. We were told that creating user stories
might be a helpful part of our development process.

\subsubsection{Action}
\begin{description}
\item[All] Produce specification and upload it to Subversion.
\end{description}


%%%%%%%%%%%%%%%%%%%%%%%%%%%%%%%%%%%%%%%%%%%%%%%%%%%%%%%%%%%%%%%%%%%%%

\subsection*{Meeting 3}
\date{3 February 2011}

\subsubsection{Present}
\begin{enumerate}
\item Katrina Samperi (Demonstrator)
\item Charlie Horrell (Secretary)
\item Yukun Wang
\end{enumerate}

\subsubsection{Apologies}
\begin{enumerate}
\item Jeremiah Via (\emph{Ill})
\item Joss Greenaway (\emph{Ill})
\end{enumerate}

\subsubsection{Summary}
During this meeting we were given general feedback on our
specifications. A major issue raised was the standard of scientific
and formal English as well as issues with grammar. We were informed
that unless this was improved in the final report it could loose us
significant marks. Also raised was the lack of updating the weekly
work logs. It was stressed these should be updated every week before
the meeting with Katrina. Some clarification was required for Yukun on the
non use of the waterfall model as we are using an agile approach. We
were also told to finally decided which networking strategy we were
going to use.  Otherwise feedback on the specification was generally
positive. We were also encouraged to produce a weekly report for all
future demonstrator meetings.

I raised the issue of the lack of details on the subversion
assessment. We were advised as we had well over 60 revisions already we
did not need to worry about it for now.


\subsubsection{Action}
\emph{None}.

%%%%%%%%%%%%%%%%%%%%%%%%%%%%%%%%%%%%%%%%%%%%%%%%%%%%%%%%%%%%%%%%%%%%%

\subsection*{Meeting 4}
\date{10  February 2011}

\subsubsection{Present}
\begin{enumerate}
\item Katrina Samperi (Demonstrator)
\item Joss Greenaway (Team Leader)
\item Charlie Horrell (Secretary)
\item Jeremiah Via
\item Yukun Wang
\end{enumerate}

\subsubsection{Summary}
During this meeting we presented to Katrina the weekly report that had
been promised at the previous weeks meeting. Feedback from this was
generally positive however Katrina requested that it would have been more
useful if the weekly report had been submitted to subversion earlier
in the day, unfortunately a large part of our timetabled man hours for
the project takes place on the Thursdays and so any report submitted
earlier in the day would likely be a false state of affairs. Katrina
approved of our production of long term goals and organization in
approaching problems.

We were also introduced to the planning game which involved different
team members approximating the amount of time taken for particular
tasks in arbitrary units consisting of the time taken for a simple
feature such as a button to be programmed in java. From this we
decided that the client server would take the majority of the weeks
ahead and as such the team should pitch in to help Jeremiah with the
production of this. We also rated the scrolling problem as taking a
long time to solve.

\subsubsection{Action}
\begin{description}
\item[Jeremiah] Networking.
\item[Charlie] AI player.
\item[Joss] Fix angles of expelled matter.
\item[Yukun] Implement scrolling.
\end{description}

%%%%%%%%%%%%%%%%%%%%%%%%%%%%%%%%%%%%%%%%%%%%%%%%%%%%%%%%%%%%%%%%%%%%%

\subsection*{Meeting 5}
\date{17 February 2011}

\subsubsection{Present}
\begin{enumerate}
\item Katrina Samperi (Demonstrator)
\item Joss Greenaway (Team Leader)
\item Jeremiah Via
\item Yukun Wang
\end{enumerate}

\subsubsection{Apologies}
\begin{enumerate}
\item Charlie Horrell (\emph{Ill})
\end{enumerate}

\subsubsection{Summary}
During this meeting the testing vivas took place. Charlie was ill so
his took place the following week.

\subsubsection{Action}
\emph{None}.

%%%%%%%%%%%%%%%%%%%%%%%%%%%%%%%%%%%%%%%%%%%%%%%%%%%%%%%%%%%%%%%%%%%%%

\subsection*{Meeting 6}
\date{24 February 2011}

\subsubsection{Present}
\begin{enumerate}
\item Katrina Samperi (Demonstrator)
\item Charlie Horrell (Secretary)
\item Jeremiah Via
\item Yukun Wang
\end{enumerate}

\subsubsection{Absent}
\begin{enumerate}
  \item Joss Greenaway
\end{enumerate}

\subsubsection{Summary}
During this meeting we had our prototype demonstration on the
project. We received largely negative feedback that our development
was stagnant, our game was buggy and that we needed multiple levels in
the single player game mode.

\subsubsection{Action}
\begin{description}
\item[All] Finish work on assigned tasks.
\end{description}

%%%%%%%%%%%%%%%%%%%%%%%%%%%%%%%%%%%%%%%%%%%%%%%%%%%%%%%%%%%%%%%%%%%%%

\subsection*{Meeting 7}
\date{3 March 2011}

\subsubsection{Present}
\begin{enumerate}
\item Katrina Samperi (Demonstrator)
\item Joss Greenaway (Team Leader)
\item Charlie Horrell (Secretary)
\item Jeremiah Via
\item Yukun Wang
\end{enumerate}

\subsubsection{Summary}
During this meeting we showed Katrina our improved game with multiple
levels fixed scrolling and better AI, it is now also less buggy and
has working music. We received largely positive feedback on this
point.


\subsubsection{Action}
\emph{None}.

%%% Local Variables:
%%% mode: latex
%%% TeX-master: "../report"
%%% End:

% LocalWords:  Osmos Darkmatter JUnit Greenaway Horrell Joss Samperi
% LocalWords:  Facebook PDF
       %% DONE(1)

\bibliographystyle{agsm}
\bibliography{report}

% Acknowledge any code you have used (if any), and also any that was
% generated automatically, e.g. by wizards.
%
% Give correct and complete bibliographic information for any sources
% cited. See the local referencing guide. For instance cite which
% books on software engineering you have used, for
% instance~\cite{software-design}. Use the \texttt{report.bib} file to
% manage your citations.
%
% When you \emph{quote} material from other sources, you must be
% absolutely clear \emph{at the point where you quote it} exactly
% which of your material is quoted and what the source is. As
% explained in \cite{SoCS:plagiarism},
%
% ``Direct quotation is not particularly common in scientific writing,
% as it is generally not the words that matter, but the meaning.
% Normally it is preferable to rewrite someone else's ideas in your
% own words, often changing the terminology and other superficial
% details to suit the new context.
%
% However, in circumstances where it is appropriate to make direct use
% of the words of another person, those words should normally be
% included within quotation marks and a reference to the source of the
% words given in the usual way.''

\end{document}

%% LocalWords: Osmos gameplay javadoc Sorge Vickers tex agsm TODO Yukun

%%% Local Variables:
%%% mode: latex
%%% TeX-master: t
%%% End:


% LocalWords:  multiplayer Greenaway Joss Horrell colorlinks
