
\chapter{Introduction}
\label{cha:introduction}

Give a brief overview and guide the reader to the important points
in the remaining sections.

This template for your report also contains some examples of how to use some
{\LaTeX} elements and commands. In particular, there are examples for tables,
how to include figures, and various environments for bullet points or
enumerations.

For further information (and here is how to do a bullet list):
\begin{itemize}
\item look on the Team Java web page,\\
\url{http://www.cs.bham.ac.uk/internal/courses/team-java/current}\\
  (Click on ``Guidance''.)
\item google ``latex''
\item look at the \LaTeX book \cite{latex}
\end{itemize}


This is how to do numbered lists:
\begin{enumerate}
\item First point
\item Second point
\item \ldots
\end{enumerate}

This is how to do sections:

\section{Some topic}\label{some}

\section{Another topic}\label{another}

If you set a label with the \texttt{label} command, you can then use
the \texttt{ref} command to refer to that section -- e.g.
section~\ref{some}. This means you don't need to know about section
numbers. Note that \texttt{\~} means a space that cannot be broken
across lines.
