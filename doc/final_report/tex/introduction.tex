
\chapter{Introduction}
\label{cha:introduction}

% Give a brief overview and guide the reader to the important points
% in the following sections.  Tell your reader what you are going to
% tell them. Overview the project highlighting the issues you
% addressed and hinting at how you solved them. The reader should be
% able to decide from the introduction what parts of the report are of
% interest to them

This report aims to document how our team was able to create a
game. We had roughly 10 weeks to create a game from start to
finish. This was a new experience for all of us and we learned many
lessons from it. We are all better programmers now and we have a
better idea of what it means to create software with a team.

We will begin by describing the final specification of our game. It
changed from our initial specification as we learned more about game
programming. This was a new domain for all of us so our initial ideas
of what it meant to make a game morphed as we gained experience in the
field. We will also explain why we chose the features we did and how
those features positively affect the user experience.

After going over the specification, we will go into the design of our
game. This section will go into the vital details of our game
architecture. You will learn about problems we encountered when
implementing features and how we solved them. We want to not only tell
you what architectural choices we made but why we made them. This will
give you insight into our problem solving abilities as a team.

We will then tell you about how we tested our game and validated its
user experience. This is an important section and we aim to highlight
how extreme programming methodologies allowed us to create good
software. We will show you how unit testing allowed us to assert our
intended uses of the code and ensure future feature enhancements would
be prevented from breaking old, stable code.

With testing and validation out of the way, we will talk about how we
managed our project. Because this was our first experience writing
code as a team, it presented us with new opportunities for
learning. We will discuss how extreme programming helped us maintain
forward momentum on our game and how utilising pair programming
ensured that everyone understood the code to a sufficient level.

We would like to wrap our report up with a general discussion of how
the module was for us. Programming a game from start to finish as a
team was a novel experience. As a result of this we made mistakes. We
will talk about the mistakes we made as a team and individually and
the lessons we learned from them. Because of this module we have an
increased maturity in our programming, meaning that we can recognise
the long-term issues with the choices we make.


\section{Initial Ideas}


At our first meeting we considered various games and discussed many
ideas. We debated the merits of various games and ended up deciding on
criterion for a good game. The game we would make would need to lend
itself to an iterative release schedule. This was important so that we
could add new features each week. We wanted to avoid anything that
would require multiple weeks to implement. We also wanted a game that
would be fun to play. This was a challenge because many entertaining
games have a lot of complex features. These two criterion constrained
our search for a game to make.


Ideas for games that were ultimately rejected included:

\begin{description}
\item [Fruit Ninja] A game which involved the slicing of fruit with a
  blade controlled by the mouse. This game was rejected because we
  could add nothing original to it or come up with any particularly
  fun or innovating multiplayer modes other than a who can post the
  highest score style game.

\item [Tower Defence] A game involving the placing of towers and
  weapons to destroy enemies that work their way through a level. This
  could possibly be made multiplayer in a cooperative mode.

\item [Zombie Tower Defence] As above but involving zombies. We
  rejected both the tower defence games as we felt they had been done
  many times before and lacked originality. There were also free
  versions available of the game online that were very playable.

\item [Space Racing Game] A racing game which takes place in space
  where the players must race while avoiding planets and
  asteroids. This was rejected again for lack of originality.
\end{description}


The game we finally settled on was a multiplayer version of Osmos
developed by Hemisphere games \cite{osmos}. We called our game
Darkmatter as we would be setting it in space with the concept of
stars rather than on the cellular level like Osmos.

All of us were happy with this selection as it included features that
played to the strengths of all members of the team. Joss with the
potential to produce exciting graphics and audio effects, Charlie with
the chance to use his physics knowledge and Jeremiah and Yukun to
utilise their strong programming skills.

%%% Local Variables: 
%%% mode: latex
%%% TeX-master: "../report"
%%% End: 

%  LocalWords:  Osmos Darkmatter Joss morphed multiplayer Yukun
