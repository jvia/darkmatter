\chapter{Discussion}
\label{chap:discussion}

% How did the team operate? What did you do to manage the team?  How
% did you resolve issues in the team?  Did you seek assistance of
% others?  Would you do anything differently in the future?  What is
% your evidence of good or bad team work
%
%
% The documentation of the team management in the report
% shows that the team endeavoured to cooperate and/or when
% issues within the team arose, they were resolved to enable to
% team to function as a coherent unit.
% - Discussion shows that the issues were addressed and provides
% evidence of what might be done differently on future projects.
% - The report acknowledges the assistance of others to resolve
% issues if necessary.
% - A work breakdown is included in the report and is supported by
% other documentation.
%

This was the first major programming project any of us had ever
undertaken in a team. As with all new experiences, we made a number of
mistakes but learnt a great deal from them. This chapter will detail
some of our shortcomings, both personally and as a team, and the
lessons we learned from this whole experience. We are all better
programmers and teammates because of this module and we feel grateful
for this opportunity. We did not follow all the best practises for
working in a team but did an admirable job nonetheless. Each of us
failed in some capacity but we all learnt from it. What follows is the
result of a lot of self-critical reflection. This section is not meant
to give the impression that we were a terrible team, in fact we all
still like each other, just that we made a number of rookie mistakes
which did have \emph{some} impact. Overall we worked well as a team
and we are all proud of the work we created together.

\section{Team Mistakes}

During the development of our game we made a series of missteps which
hindered our development process. None of the mistake were egregious
or even overt at the time but after spending some time to reflect on
the module, we have been able to recognise some ways in which we could
have done better. Among our mistakes were a lack of communication,
ambiguity in responsibility for a particular feature and the failure
of completing a task on time.

We were plagued by communication issues throughout the project. Our
team used Facebook as a means of communication and in retrospect it
was a mistake. Our group Facebook wall is almost entirely populated by
the communication of one person. Facebook is a passive stream of
consumption, i.e., users do not respond to most input they receive on
it. This pattern of use does not lend itself well to a team project
where input from all members is important. As a result of using
Facebook, one person would post questions but never receive
responses. This got in the way of good communication and hurt team
morale. On the next project, alternative means of communication will
be used.

Another mistake was the lack of a rigid break up of work requirements.
This was in part to our agile philosophy of team code ownership and
the responsibility of everyone to work on every class. It, however,
resulted in a lack of understanding about who had to produce an aspect
of the software. It allowed us to be apathetic about the project as a
whole at times, such as week six when the majority of the team
prioritised the Software System Components assignment. In the future,
stricter assignment of work would be beneficial.

It is worth noting that our team leader Joss disappeared midway
through term. Obviously, losing a team member has a negative impact on
a team's ability to complete work. It was also a confusion situation
because we never heard anything from him on the matter. He just
disappeared and we had to find out through his friends that he had
become ill. The main problem from this situation was that our team was
left without a team leader. No one ever took the place of team leader
and so our development happened in a more ad hoc manner. This is
something we should have immediately rectified by chasing him down on
his university email and reassigning the position when it became clear
he would not have been returning.

Our team was also negatively impacted by the failure of some members
to complete their assigned tasks on time. While this was rare, it did
mean that others had to pick up the slack and start working on parts
of the code that they were not meant to be working on. Having to do
another's work prevented some features from making their way into the
game. This is an issue which we should have addressed at the
beginning. We had trouble with this because it is hard to admonish
people we like. We have all learnt the importance of ensuring that
team members complete their task. In future projects, we will all be
more forward about the failures of our teammates to complete their
work.

\section{Individual Mistakes}

\subsection{Charles}

This was my first time working in a team in a software development
environment. While familiar with the processes involved in large team
projects academically, I was less prepared for how this would relate
to a software project. I made quite a few mistakes over the period
which I have managed to learn from.

My main mistake was not maintaining a constant work rate throughout
the term. I prioritised some other modules above this module and
particularly around the time I was doing my communication skills
presentations and essays I was not able to offer the input to the team
java module that I should have. This is a valuable lesson for time
management and work planning. In the future I will plan for these
thing and anticipate it and make it up to my team by doing more work
in advance of the time I am going to be working on other projects.

Another mistake I made was not getting as involved in the production
of software as either Yukun or Jeremiah. While I did write code for
the \texttt{VelocityVector} class, the AI player, unit tests and
produce some levels I did not really write enough code overall. This
was because I took my role as team secretary too literally and perhaps
put too much emphasis of my time on producing team meeting minutes,
weekly reports, user stories and working on the final report than
writing code. In future projects I will work on writing more code.

Another mistake I felt I made was not asking Jeremiah, the most
experienced coder on our team, for help with writing more code. I was
afraid that my inadequacies as a coder would show through and I was
embarrassed by this. I should have utilised the pair programming
sessions more fully in the later weeks of the project as in the first
six weeks to help me code and learn from these two excellent coders. I
learnt from this to always make use of such excellent resources in
future projects.

\subsection{Jeremiah}

While I feel I worked well on a team, I did make a number of mistakes
over the course of our project. These were lack of leadership, the
inability to correctly deal with a teammate's missed deadline and
difficult in sharing code.

In retrospect I should have taken the leadership position of our
team. I had spent the previous term reading books on teamwork and
wanted to apply some of what I had learnt. I had heard of Joss'
reputation for partying and disregard for work and I had hoped to stem
this behaviour. So I applied a technique I had read in a book by
\cite{carnegie1936}: ``Give the other person a fine reputation to live
up to.'' So I figured by making Joss team leader he would live up to
the leadership reputation. It did in fact work for a good portion of
the term. But, unfortunately he disappeared and never gave us any
indication of what had happened to him. I should have taken his role
as soon as a week had passed. This would have prevented our team from
remaining in decision paralysis for son long.

Another mistake I made was not properly dealing with other team
member's uncompleted work correctly. I did not do a good enough job of
chasing them down and making sure they completed their task; getting
the demonstrator involved if nothing had changed. Instead, I would
take it upon myself to finish their task for them. This helps no one
on the team. It delays progress because my time is spent doing other
people's work and it give the other person a pass when they need to
understand that their contribution is important. In the future, I will
make an effort to make sure all my team members contribute evenly.

I also had a hard time sharing code. When programming, I tended to be
pretty controlling about the way the code would function. This made
pair programming sessions frustrating because I wanted to take control
and immediately begin typing what I felt was the correct way to
proceed. Overall, the pair programming sessions did help me with this
feeling. I am now a lot more open minded and patient when it comes to
coding with others.

\subsection{Yukun}

This is my first time to do a team software project. Although the
final result looks pretty good, I still made some mistake in the
process.

The main thing is team work. To be honest, in this project we worked
in a team, but not as a team. The team members worked separately
although at beginning we planned pair programming. And when we lost
the team leader, we did not point out a temporary leader which is one
of the reasons why our project was so slow.

Secondly, I did not have a good communication with the other team
members. My poor English made me ashamed to talk and explain my
ideas. And then since we had the whole term time to finish this
project, I let it fall behind. Especially in the middle term, my
schedule got bogged down which made me less enthusiastic about
coding. Another mistake I made in the project was not fully utilizing
the Junit test. Due to a lack of experience, I always wrote JUnit
tests after programming instead of before. So when I was coding, I was
confused about the target of each method.

In this project, I learned a lot things I never learned before. I
learned how to user version control software by using subversion. And
thanks to Jeremiah, we had Maven which made our work process
convenient. Meanwhile, I learned how to use JUnit test to set goals
for programs. On the other hand, team work is not as simple as I
thought. I should communicate better with team members and when a
problem occurs the team needs to make a decision in time.

%%% Local Variables:
%%% mode: latex
%%% TeX-master: "../report"
%%% End:

% LocalWords:  Facebook Joss hoc another's JUnit

