\chapter{Conclusions}
\label{chap:conclusions}

% Evaluate what you have achieved in your project in objective terms
% (not just things like "we are all happy with the result."). What are
% the strengths and limitations of your project? If you had more time,
% what could you add? If you could do it all over again, what would
% you do differently? Are there general things about software or team
% management that you have learnt in this project which you could
% apply if you were going to work on a (perhaps completely different)
% team project in the future?
%
% Tell your reader what you have told them. You don't want the reader
% to forget what your key points and findings are. You want them to go
% away convinced that you have done a good job and can argue for your
% project.

Over the course of the past term, our team implemented an exciting
game from start to finish. Doing this exposed all of us to new facets
of software development and teamwork. We are all grateful for the
experience and we will be able to apply the these lessons to future
software projects.

We talked about our specification. It changed as we gained more
knowledge in the domain of games development. As time went on, our
specification became more exacting until it turned into what it is
now. Our specification talked about features we chose for the game and
why we felt they were good for the user experience.

The next chapter talked about our design. This space was used to talk
about the design decisions when made and \emph{why} we made them. We
talked at a high level of abstraction about important components of
the overall system architecture, most notably the networking
architecture.

After the design we spoke at length about our testing strategy. We had
good test coverage and this helped us know our game was working
correctly. It also gave us the confidence to implement new features
without breaking old ones.

In the discussion, we spoke about our failings as a team and
individually. None of these failings were enough to warrant
demonstrator involvement nor was their impact too great a burden for
us to create the great game we did. This section was an opportunity
for each of us to share ways in which we failed and the lessons we
learned. 

This project was a great opportunity for each of us to become better
programmers, problem solvers and teammates.  We have learned skills
ranging from version control to the planning game that will ensure
that we will be more effective on our next project. Each of us are now
more equipped to handle big projects in the future either alone or on
a team.

%%% Local Variables: 
%%% mode: latex
%%% TeX-master: "../report"
%%% End: 
