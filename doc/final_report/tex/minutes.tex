\chapter{Meeting Minutes}
\label{cha:minutes}

\section{Group Meetings}

\subsection*{Meeting 1}
\date{20 January 2011}

\subsubsection{Present}
\begin{enumerate}
\item Joss Greenaway (Team Leader)
\item Charlie Horrell (Secretary)
\item Jeremiah Via
\item Yukun Wang
\end{enumerate}

\subsubsection{Summary}

During this meeting we spent some time getting to know each other and
confirmed the appointment of Joss and myself as officers. The main
content of the meeting was talking about possible game ideas. Ideas
proposed included:
\begin{itemize}
\item An physics game styled on Osmos with added features and
  originality. We decided this was our primary idea and Joss came up
  with the project name of ``Darkmatter'' for the game.
\item An Osmos style racing game using the Osmos propulsion mechanics
  to race blobs around an obstacle course.
\item A tower defense style game possibly involving zombies and a
  cooperative mode.
\item Fruit ninja.
\item Fall down: a two player game involved in a race to see which
  player can fall down quickest.
\end{itemize}


Once we had decided on Osmos as our primary idea and zombie tower
defense as our fall back, we considered ways to put our own spin on
Osmos. These included the addition of weapons to the right click, such
as an antimatter weapon that would split any other dark matter stars
in half or reduce size. And also a weapon to stop the balls in full
flow. The main aspect of originality that we add as well as setting it
in space is the addition of cooperative and competitive game play. We
talked about various modes such as straight up death match style,
i.e., just become the biggest player, to other game modes that would
require players to work together to take down a larger enemy or maybe
a mode where the players are segregated and need to complete tasks
with each other to complete the level.

We also touched on the software engineering principles we would employ
and this included testing methods such as JUnit and the possible use
of pair programming was mentioned as a method as well as other methods
such as ``Scrum''. Jeremiah suggested the use of Maven for our project.

\subsubsection{Action}

\begin{description}
\item[All] Research Osmos and Subversion
\end{description}

%%%%%%%%%%%%%%%%%%%%%%%%%%%%%%%%%%%%%%%%%%%%%%%%%%%%%%%%%%%%%%%%%%%%%

\subsection*{Meeting 2}
\date{25 January 2011}

\subsubsection{Present}
\begin{enumerate}
\item Joss Greenaway (Team Leader)
\item Charlie Horrell (Secretary)
\item Jeremiah Via
\item Yukun Wang
\end{enumerate}

\subsubsection{Summary}

We met for the second time to discuss coding practices and the
structure of our game. We decided that we would leave the start of
coding till week 3 when we knew how to work sockets and networking
into the code as we felt that this would impact our code at every
level. We decided on a pair programming schedule where we would each
code with each other member of the team for an hour a week. We also
decided that we would have this meeting every week. After these
discussions and some discussion of the physics engine we decided to
head to the labs to get Subversion working and test it out. Jeremiah
helped the team with this as he has prior experience of using
subversion.

\subsubsection{Action}
\begin{description}
\item[All] Research coding structure and Osmos. Submit progress logs
  to Subversion.
\item[Charlie] Produce timetable of pair programming and other Team
  Java commitments and upload to Subversion.
\end{description}

%%%%%%%%%%%%%%%%%%%%%%%%%%%%%%%%%%%%%%%%%%%%%%%%%%%%%%%%%%%%%%%%%%%%%

\subsection*{Meeting 3}
\date{1 February 2011}

\subsubsection{Present}
\begin{enumerate}
\item Joss Greenaway (Team Leader)
\item Charlie Horrell (Secretary)
\item Jeremiah Via
\item Yukun Wang
\end{enumerate}


\subsubsection{Summary}

We decided to meet to discuss the impending deadline for the
specification and to make sure we were broadly on the same page
regarding what to put in it. We also talked over software engineering
principles that we were going to use including maven and testing. We
also discussed the GUI and the possible methods of animating the dark
matter. Yukun also showed the team a small model he had made with an
object bouncing around a screen and interacting with the edges. We
then created a sandbox in the subversion repository in which we could
put experimental code to play around with. We decided on physics that
would go into the game that included making the game universe
frictionless to make the modeling easier and to replicate space.

\subsubsection{Action}
\begin{description}
\item[All] Produce specification and submit to Subversion.
\end{description}

%%%%%%%%%%%%%%%%%%%%%%%%%%%%%%%%%%%%%%%%%%%%%%%%%%%%%%%%%%%%%%%%%%%%%

\subsection*{Meeting 4}
\date{1 February 2011}

\subsubsection{Present}
\begin{enumerate}
\item Joss Greenaway (Team Leader)
\item Charlie Horrell (Secretary)
\item Jeremiah Via
\item Yukun Wang
\end{enumerate}


\subsubsection{Summary}

We discussed how we were going to approach the pair programming
sessions over the coming week. It was decided upon that for the
sessions scheduled straight after the meeting that Jeremiah and Joss
would work on the matter class and charlie and Yukun would work on the
physics class. This work was then undertaken and submitted to
subversion. It was raised that charlie and Yukun were not uploading
comments when they submitted work to subversion and this should be
remedied. Jeremiah and Yukun also showed of various prototypes that
they had uploaded to subversion to test out aspects of the game.

\subsubsection{Action}

\begin{description}
\item[Joss \& Jeremiah] Produce Matter Class.
\item[Charlie \& Yukun] Produce Physics Class.
\end{description}

%%%%%%%%%%%%%%%%%%%%%%%%%%%%%%%%%%%%%%%%%%%%%%%%%%%%%%%%%%%%%%%%%%%%%

\subsection*{Meeting 5}
\date{8 February 2011}

\subsubsection{Present}
\begin{enumerate}
\item Joss Greenaway (Team Leader)
\item Charlie Horrell (Secretary)
\item Jeremiah Via
\item Yukun Wang
\end{enumerate}


\subsubsection{Summary}

Firstly we discussed the progress of the previous and team members
were bought up to date on the demonstrator meeting from the previous
week. We discussed the points raised in that meeting and endeavored to
address the issues that came up such as using more scientific English
and producing a detailed weekly report. It was decided that charlie
would produce this in time for the demonstrator meeting for the
following weeks. Jeremiah demonstrated to us the Darkmatter class he
had written and Joss showed us the draft of an opening video for the
game. We then decided on tasks for the week and ranked tasks that
needed to be done in order of priority. The main issues to be
undertaken were programming an AI for the other player in the single
player mode, start programming the networking, fixing the ejection
angles of the balls of matter and introduce an egocentric scrolling
view.

\subsubsection{Action}

\begin{description}
\item[Joss] Fix ejection angles of matter.
\item[Charlie] Produce AI for other player.
\item[Jeremiah] Start the client-server networking.
\item[Yukun] Develop a scrolling system with the player always at the center.
\end{description}

%%%%%%%%%%%%%%%%%%%%%%%%%%%%%%%%%%%%%%%%%%%%%%%%%%%%%%%%%%%%%%%%%%%%%

\subsection*{Meeting 6}
\date{15 February 2011}

\subsubsection{Present}
\begin{enumerate}
\item Joss Greenaway (Team Leader)
\item Charlie Horrell (Secretary)
\item Jeremiah Via
\item Yukun Wang
\end{enumerate}


\subsubsection{Summary}

This was an extremely brief meeting where we just decided that we
would carry on with the previous weeks goals and that Joss would work
on producing the media player. Jeremiah also filled us in on how the
changes to the client-server would effect the rest of the game code
and helped us install the latest versions of maven on our machines.

\subsubsection{Action}
\begin{description}
\item[Joss] Produce media player for the game.
\item[Charlie] Produce AI for other player.
\item[Jeremiah] Start the client-server networking.
\item[Yukun] Develop a scrolling system with the player always at the
  center.
\end{description}

%%%%%%%%%%%%%%%%%%%%%%%%%%%%%%%%%%%%%%%%%%%%%%%%%%%%%%%%%%%%%%%%%%%%%

\subsection*{Meeting 7}
\date{22 February 2011}

\subsubsection{Present}
\begin{enumerate}
\item Charlie Horrell (Secretary)
\item Jeremiah Via
\item Yukun Wang
\end{enumerate}


\subsubsection{Summary}

This was another brief meeting where we again decided to carry on with
the previous weeks work as it was still in progress. Also tidy up and
review JUnit tests for the testing vivas later on in the week.

\subsubsection{Action}
\begin{description}

\item[Joss] Produce media player for game.
\item[Charlie] Produce AI for other player.
\item[Yukun] Develop an scrolling system with the player always at the center.
\item[Jeremiah] Start the client server networking.
\item[All] Review tests.
\end{description}


%%%%%%%%%%%%%%%%%%%%%%%%%%%%%%%%%%%%%%%%%%%%%%%%%%%%%%%%%%%%%%%%%%%%%

\subsection*{Meeting 8}
\date{1 March 2011}

\subsubsection{Present}
\begin{enumerate}
\item Joss Greenaway (Team Leader)
\item Charlie Horrell (Secretary)
\item Jeremiah Via
\item Yukun Wang
\end{enumerate}


\subsubsection{Summary}

This meeting we talked frankly about our disappointing progress of the
previous few weeks and decided on a course of action that would propel
us forward. This involved finishing off previous tasks including
working on finishing networking and making AI smarter and scrolling
smoother. We also decided that a level builder would be a good feature
to improve the games single player mode.

\subsubsection{Action}
\begin{description}

\item[Charlie] Keep working on AI; begin working on final report.
\item[Joss] Continue working on media player.
\item[Jeremiah] Work on networking.
\item[Yukun] Create a level builder.
\end{description}


%%%%%%%%%%%%%%%%%%%%%%%%%%%%%%%%%%%%%%%%%%%%%%%%%%%%%%%%%%%%%%%%%%%%%

\subsection*{Meeting 9}
\date{8 March 2011}

\subsubsection{Present}
\begin{enumerate}
\item Joss Greenaway (Team Leader)
\item Charlie Horrell (Secretary)
\item Jeremiah Via
\item Yukun Wang
\end{enumerate}


\subsubsection{Summary}

This meeting we talked about the things we needed to do to polish off
our project. This included working more on the final report,
integrating the networking and producing a menu to polish up the game.

\subsubsection{Action}

\begin{description}
\item[Jeremiah] Continue working on networking.
\item[Charlie] Work on final report and submit an iteration to Subversion.
\item[Yukun] Work on menu.
\item[Joss] Work on media player.
\end{description}

%%%%%%%%%%%%%%%%%%%%%%%%%%%%%%%%%%%%%%%%%%%%%%%%%%%%%%%%%%%%%%%%%%%%%

% \subsection*{Meeting 10}
% \date{25 January 2011}
%
% \subsubsection{Present}
% \begin{enumerate}
% \item Joss Greenaway (Team Leader)
% \item Charlie Horrell (Secretary)
% \item Jeremiah Via
% \item Yukun Wang
% \end{enumerate}
%
%
% \subsubsection{Summary}
% \subsubsection{Action}


\section{Demonstrator Meetings}
\subsection*{Meeting 1}
\date{20 January 2011}

\subsubsection{Present}
\begin{enumerate}
\item Katrina Samperi (Demonstrator)
\item Joss Greenaway (Team Leader)
\item Charlie Horrell (Secretary)
\item Jeremiah Via
\item Yukun Wang
\end{enumerate}

\subsubsection{Summary}
During this meeting we introduced ourselves to Katrina and pitched our
idea of Darkmatter to her. Katrina approved our idea and informed us of
the basic structure of the module and the importance of
logs,subversion and team work. She also prompted us into choosing a
team name! After a brief discussion about Pokemon we settled on team
``Giant Cow'' because to quote Joss ``one of the new Pokemon looks just
like a giant cow.'' The name was catchy yet suitably absurd and
humorous so ``Giant Cow Games'' was born! We also asked Katrina about the
specification that is due in by the end of week two and it was
suggested to us that we should not go into technical detail but give a
more general idea of what our game will be about and the principles
used to code it. We were also warned about pitfalls previous teams had
had in splitting up components of the project arbitrarily and then
meeting at a later date to merge sections, as this had no fail safe
and often required a lot of extra work to integrate components. We
also raised the issue of possibly using games libraries or other
source code in the programming of the game. Katrina said she would look
into this for us and let us know. Katrina also indicated to us that pair
programming was an excellent idea.

\subsubsection{Action}
\begin{description}
\item[All] Research Osmos and Subversion. Think about content for
  specification.
\item[Charlie] Update Facebook group, minutes and arrange further
  meetings.
\end{description}


%%%%%%%%%%%%%%%%%%%%%%%%%%%%%%%%%%%%%%%%%%%%%%%%%%%%%%%%%%%%%%%%%%%%%

\subsection*{Meeting 2}
\date{7 January 2011}

\subsubsection{Present}
\begin{enumerate}
\item Katrina Samperi (Demonstrator)
\item Joss Greenaway (Team Leader)
\item Charlie Horrell (Secretary)
\item Jeremiah Via
\item Yukun Wang
\end{enumerate}

\subsubsection{Summary}
During this meeting we were given more precise instructions on when to
submit the specification. We were told it had to be in for 12pm on
subversion in PDF format. We also talked about our progress for the
week, including pair programming and testing with maven. Katrina approved
of these measures. We also decided to drop scrum programming partly
because of the roles that different people would have to take on made
the process seem convoluted. We were told that creating user stories
might be a helpful part of our development process.

\subsubsection{Action}
\begin{description}
\item[All] Produce specification and upload it to Subversion.
\end{description}


%%%%%%%%%%%%%%%%%%%%%%%%%%%%%%%%%%%%%%%%%%%%%%%%%%%%%%%%%%%%%%%%%%%%%

\subsection*{Meeting 3}
\date{3 February 2011}

\subsubsection{Present}
\begin{enumerate}
\item Katrina Samperi (Demonstrator)
\item Charlie Horrell (Secretary)
\item Yukun Wang
\end{enumerate}

\subsubsection{Apologies}
\begin{enumerate}
\item Jeremiah Via (\emph{Ill})
\item Joss Greenaway (\emph{Ill})
\end{enumerate}

\subsubsection{Summary}
During this meeting we were given general feedback on our
specifications. A major issue raised was the standard of scientific
and formal English as well as issues with grammar. We were informed
that unless this was improved in the final report it could loose us
significant marks. Also raised was the lack of updating the weekly
work logs. It was stressed these should be updated every week before
the meeting with Katrina. Some clarification was required for Yukun on the
non use of the waterfall model as we are using an agile approach. We
were also told to finally decided which networking strategy we were
going to use.  Otherwise feedback on the specification was generally
positive. We were also encouraged to produce a weekly report for all
future demonstrator meetings.

I raised the issue of the lack of details on the subversion
assessment. We were advised as we had well over 60 revisions already we
did not need to worry about it for now.


\subsubsection{Action}
\emph{None}.

%%%%%%%%%%%%%%%%%%%%%%%%%%%%%%%%%%%%%%%%%%%%%%%%%%%%%%%%%%%%%%%%%%%%%

\subsection*{Meeting 4}
\date{10  February 2011}

\subsubsection{Present}
\begin{enumerate}
\item Katrina Samperi (Demonstrator)
\item Joss Greenaway (Team Leader)
\item Charlie Horrell (Secretary)
\item Jeremiah Via
\item Yukun Wang
\end{enumerate}

\subsubsection{Summary}
During this meeting we presented to Katrina the weekly report that had
been promised at the previous weeks meeting. Feedback from this was
generally positive however Katrina requested that it would have been more
useful if the weekly report had been submitted to subversion earlier
in the day, unfortunately a large part of our timetabled man hours for
the project takes place on the Thursdays and so any report submitted
earlier in the day would likely be a false state of affairs. Katrina
approved of our production of long term goals and organization in
approaching problems.

We were also introduced to the planning game which involved different
team members approximating the amount of time taken for particular
tasks in arbitrary units consisting of the time taken for a simple
feature such as a button to be programmed in java. From this we
decided that the client server would take the majority of the weeks
ahead and as such the team should pitch in to help Jeremiah with the
production of this. We also rated the scrolling problem as taking a
long time to solve.

\subsubsection{Action}
\begin{description}
\item[Jeremiah] Networking.
\item[Charlie] AI player.
\item[Joss] Fix angles of expelled matter.
\item[Yukun] Implement scrolling.
\end{description}

%%%%%%%%%%%%%%%%%%%%%%%%%%%%%%%%%%%%%%%%%%%%%%%%%%%%%%%%%%%%%%%%%%%%%

\subsection*{Meeting 5}
\date{17 February 2011}

\subsubsection{Present}
\begin{enumerate}
\item Katrina Samperi (Demonstrator)
\item Joss Greenaway (Team Leader)
\item Jeremiah Via
\item Yukun Wang
\end{enumerate}

\subsubsection{Apologies}
\begin{enumerate}
\item Charlie Horrell (\emph{Ill})
\end{enumerate}

\subsubsection{Summary}
During this meeting the testing vivas took place. Charlie was ill so
his took place the following week.

\subsubsection{Action}
\emph{None}.

%%%%%%%%%%%%%%%%%%%%%%%%%%%%%%%%%%%%%%%%%%%%%%%%%%%%%%%%%%%%%%%%%%%%%

\subsection*{Meeting 6}
\date{24 February 2011}

\subsubsection{Present}
\begin{enumerate}
\item Katrina Samperi (Demonstrator)
\item Charlie Horrell (Secretary)
\item Jeremiah Via
\item Yukun Wang
\end{enumerate}

\subsubsection{Absent}
\begin{enumerate}
  \item Joss Greenaway
\end{enumerate}

\subsubsection{Summary}
During this meeting we had our prototype demonstration on the
project. We received largely negative feedback that our development
was stagnant, our game was buggy and that we needed multiple levels in
the single player game mode.

\subsubsection{Action}
\begin{description}
\item[All] Finish work on assigned tasks.
\end{description}

%%%%%%%%%%%%%%%%%%%%%%%%%%%%%%%%%%%%%%%%%%%%%%%%%%%%%%%%%%%%%%%%%%%%%

\subsection*{Meeting 7}
\date{3 March 2011}

\subsubsection{Present}
\begin{enumerate}
\item Katrina Samperi (Demonstrator)
\item Joss Greenaway (Team Leader)
\item Charlie Horrell (Secretary)
\item Jeremiah Via
\item Yukun Wang
\end{enumerate}

\subsubsection{Summary}
During this meeting we showed Katrina our improved game with multiple
levels fixed scrolling and better AI, it is now also less buggy and
has working music. We received largely positive feedback on this
point.


\subsubsection{Action}
\emph{None}.

%%% Local Variables:
%%% mode: latex
%%% TeX-master: "../report"
%%% End:

% LocalWords:  Osmos Darkmatter JUnit Greenaway Horrell Joss Samperi
% LocalWords:  Facebook PDF
